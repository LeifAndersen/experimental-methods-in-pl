\section{Discussion}
\label{discussion}

Here we mention a few anecdotes from this semester and outline future work.

\subsection{Disabling Contracts}
\label{disabling-contracts}

The biggest challenge this semester was figuring out how to turn off contracts.
We learned that contracts are not applied and created uniformly; there are different channels for user-defined contracts, built-in contracts, typed racket contracts, and contracts provided at module boundaries.

Our first reasonable attempt at disabling contracts was to modify calls to \mono{apply-contract} within a macro for building contracts~\cite{no-contracts1}.
Unfortunately it only disabled user-defined contracts.
Later we replaced typed racket contracts~\cite{no-contracts2} and finally module-boundary contracts~\cite{no-contracts3} with trivial checks.

We are grateful to the Racket developers for helping us achieve what we have, but we are still concerned we may be missing significant contract checks.
As is, we intentionally did not remove the contract checks for primitive operations like \mono{+} because they are implemented manually in the C source.

Additionally, our solution of turning off contracts by replacing them with the trivial \mono{any/c} contract entails a few tradeoffs.
We still assume the overhead of creating the contract we eventually discard, and we likely incur a level of indirection through the \mono{any/c} contract, even though it accepts its argument instantly.\footnote{We say `likely' because we believe Racket has some optimizations to reduce contract checks. We do not know the exact nature of these optimizations, but think they might include erasing \mono{any/c} checks.}
In the future, we should investigate alternatives for disabling contracts and measure the difficulty of using option contracts~\cite{option-contracts} from the start.

\subsection{Types of Boundaries}
\label{types-of-boundaries}

The contracts we used in our Project Euler benchmarks are quite different in style from the soft contracts artifact.
Indeed, conversation with Matthias Felleisen revealed that Racket's \mono{define/contract} is rarely used by ``experienced'' programmers.
The opinion is that contracts are too expensive for functions to check, so instead contracts should only protect boundaries between ``friendly code'' and ``the outside world''.
This is exactly the object-oriented viewpoint, that internal functions may break invariants that outsiders cannot.

At any rate, it seems our style of using contracts differs from the norm.
From this we draw two conclusions: first, we should test larger programs with more boundaries to approximate contracts' practical use.
Second, we should measure the cost of \mono{define/contract} against \mono{provide/contract} to determine just how much more expensive the former is.

\subsection{Objects vs. Functions}

The Zombie game was implemented in an object-oriented style and saw markedly less improvement by the JIT than other programs.
We should determine whether this holds true for Racket classes in general\textemdash that the JIT cannot optimize object-oriented programs as effectively regardless of whether contracts are on or off.
Asumu Takikawa and Dan Feltey are already exploring the dynamics of contracts and objects, so we should ask what they have learned so far.

\subsection{Comparing Contracts}
It is possible that some types of contracts, or some styles of contract application, are more amenable to the JIT than others.
So far we have seen the JIT does better on a program without contracts, but it would be interesting to find a compelling example where adding contracts improves the number of optimizations the JIT can perform.

\subsection{More Experiments}

Now that we have an experimental setup in place and have analyzed some preliminary data, we should run and record data for more programs.
The next obvious step is to explore the Racket package archives for larger programs with contracts.
We should also test JIT performance after the VM has reached a steady state, as opposed to our strategy of starting a new VM for each trial.

