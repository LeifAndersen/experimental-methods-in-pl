\newpage
\section{Discussion}
\label{discussion}

\subsection{Disabling Contracts}
\label{disabling-contracts}

The biggest challenge this semester was figuring out how to turn off contracts.
We learned that contracts are not applied and created uniformly; there are different channels for user-defined contracts, built-in contracts, typed racket contracts, and contracts provided at module boundaries.

Our first reasonable attempt at disabling contracts was to modify calls to \mono{apply-contract} within a macro for building contracts~\cite{no-contracts1}.
Unfortunately it only disabled user-defined contracts.
Later we replaced typed racket contracts~\cite{no-contracts2} and finally module-boundary contracts~\cite{no-contracts3} with trivial checks.

We are grateful to the Racket developers for helping us achieve what we have, but we are still concerned we may be missing significant contract checks.
As is, we intentionally did not remove the contract checks for primitive operations like \mono{+} because they are implemented manually in the C source.

Additionally, our solution of turning off contracts by replacing them with the trivial \mono{any/c} contract entails a few tradeoffs.
We still assume the overhead of creating the contract we eventually discard, and we likely incur a level of indirection through the \mono{any/c} contract, even though it accepts its argument instantly.\footnote{We say `likely' because we believe Racket has some optimizations to reduce contract checks. We do not know the exact nature of these optimizations, but think they might include erasing \mono{any/c} checks.}
In the future, we should investigate alternatives for disabling contracts and measure the difficulty of using option contracts~\cite{option-contracts} from the start.
