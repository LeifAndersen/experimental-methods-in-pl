\newpage
\appendix
\section*{Appendix A: Benchmark Descriptions}
\label{appendix}

\subsection*{PE-27~\hrulefill}
\begin{description}
\item[Source:] Project Euler
\item[Lines of code:] 56
\item[Overview:] For quadratics of the form $n^2 + an + b$, where $|a| < 1000$ and $|b| < 1000$, find the product of coefficients $a$ and $b$ for which the expression produces the maximum number of primes for consecutive values of $n$, starting at $n=0$.
\item[Details:] 
  We solve this problem na\"ively by trying every combination of coefficients $a$ and $b$.
  First we use the \mono{for/list} combinator to take the cartesian product of the range $[-1000,~1000]$, then we begin at $n=0$ and check if the value of $n^2 + an + b$ is prime.
  We repeat this process for increasing values of $n$ until we find a non-prime, at which point we record the value of $n$ and compare it to previous results.
  After trying all possible pairs of coefficients $(a,b)$, the algorithm returns the pair associated with the largest value of $n$.

  Contracts surround each function in our script; that is, the \mono{main} function, the function to generate \mono{all-coefficients} within a range, the function \mono{check\_all} that searches for the best pair of coefficients, the two functions that compute a polynomial for increasing values of $n$, and finally the \mono{is\_prime} function.
  We check if a number is prime by searching a list of the first million prime numbers.
  The script uses one higher order contract: to compute the polynomial given a value for $n$.
\end{description}
\url{https://github.com/LeifAndersen/experimental-methods-in-pl/blob/master/project-euler/27.rkt}

\subsection*{PE-33~\hrulefill}
\begin{description}
\item[Source:] Project Euler
\item[Lines of code:] 83
\item[Overview:]
  Find the product of the 4 fractions less than 1 in value with two digits in the numerator and denominator that may be ``simplified'' by removing a non-zero digit $x$ from both the numerator and denominator.
  For example, $\frac{49}{98}$ may be simplified to $\frac{4}{8}$ by removing a 9 from both the numerator and denominator.
\item[Details:] 
  Again, this is a na\"ive algorithm.
  We represent a fraction as a \mono{cons} pair and search the cartesian product of $[10,~99]$ for pairs representing fractions less than 1 where removing a common digit from the string representations of the numerator and denominator gives a fraction with the same quotient (as computed by \mono{/} and \mono{equal?}).
  To remove digits, we first build a list of digits shared between the numerator and denominator, then iterate over that.
  Each non-library function is protected by a flat contract, including the \mono{str-member} and \mono{number->list} functions.
\end{description}
\url{https://github.com/LeifAndersen/experimental-methods-in-pl/blob/master/project-euler/33.rkt}

\subsection*{PE-34~\hrulefill}
\begin{description}
\item[Source:] Project Euler
\item[Lines of code:] 30
\item[Overview:]
  Find the sum of all numbers that are equal to sum of the factorials of their digits.
\item[Details:] 
  Our algorithm is designed to be simple and computationally-heavy rather than clever.
  For natural numbers less than the arbitrary ceiling of 9000000, we sum those numbers that are equal to the sum of the sum of the factorials of their digits.
  We get digits by converting a number to a string with \mono{~a}, then iterate over the string's characters.
  We compute factorial recursively with an accumulator.
  The factorial function, like the functions to iterate over natural numbers and their digits, is protected by a flat contract.
\end{description}
\url{https://github.com/LeifAndersen/experimental-methods-in-pl/blob/master/project-euler/34.rkt}

\subsection*{PE-46~\hrulefill}
\begin{description}
\item[Source:] Project Euler
\item[Lines of code:] 40
\item[Overview:]
  Find the smallest odd composite number that cannot be written as the sum of a prime number and twice a square number.
\item[Details:] 
  For odd numbers $n$ between 9 and 9,999, the algorithm finds all prime numbers $p$ less than $n$ and checks if the difference $n - p$ is square.
  Output is the \mono{for/or} disjunction over all valid combinations of odd composites and primes.
  We pre-compute prime numbers below 10,000 via the Sieve of Eratosthenes and save them in a stream.
  The functions to generate prime numbers and to try decomposing odd composites into a prime and a square are guarded by contracts.
  All contracts are flat except for the contract on \mono{stream-filter}, the function used to get primes below our current $n$.
\end{description}
\url{https://github.com/LeifAndersen/experimental-methods-in-pl/blob/master/project-euler/46.rkt}

\subsection*{Zombie~\hrulefill}
\begin{description}
\item[Source:] Soft Contracts
\item[Lines of code:] 313
\item[Overview:]
  Runs a pre-defined set of moves on a game of tag.
\item[Details:]
  Zombie is written in object-oriented style and uses a few higher-order contracts to guarantee function arguments have a certain object structure.
  These are \mono{posn/c}, \mono{player/c}, \mono{zombie/c}, \mono{horde/c}, \mono{zombies/c}, and \mono{world/c}.
  Thirteen of the games fifteen exports are contracted using only these higher-order contracts.
\end{description}
\url{https://github.com/LeifAndersen/experimental-methods-in-pl/blob/master/soft-contract/zombie.rkt}

\subsection*{Snake~\hrulefill}
\begin{description}
\item[Source:] Soft Contracts
\item[Lines of code:] 312
\item[Overview:]
  Runs a pre-defined set of moves navigating a snake across a board of death traps and apples.
\item[Details:] 
  Snake is built from 9 different modules, each protected from external modules by contracts.
  These are \mono{image}, \mono{data}, \mono{const}, \mono{collide}, \mono{cut-tail}, \mono{motion-help}, \mono{motion}, \mono{handlers}, and \mono{scenes}.
  Functions within these modules are not contracted, nevertheless the high granularity of the module architecture leads to high contract overhead.
\end{description}
\url{https://github.com/LeifAndersen/experimental-methods-in-pl/blob/master/soft-contract/snake.rkt}

\subsection*{Tetris~\hrulefill}
\begin{description}
\item[Source:] Soft Contracts
\item[Lines of code:] 433
\item[Overview:]
  Runs a pre-defined set of moves on a game of Tetris.
\item[Details:] 
  Like Snake, the Tetris game is built of a many small modules.
  There are 11 all together: \mono{data}, \mono{consts}, \mono{block}, \mono{image}, \mono{list-fun}, \mono{bset}, \mono{elim}, \mono{tetras}, \mono{aux}, \mono{world}, and \mono{visual}.
  These are again flat contracts, but having many small modules leads to a high contracts overhead.
\end{description}
\url{https://github.com/LeifAndersen/experimental-methods-in-pl/blob/master/soft-contract/tetris.rkt}

\subsection*{Matrix~\hrulefill}
\begin{description}
\item[Source:] Racket Documentation
\item[Lines of code:] 8
\item[Overview:]
  Multiplies two randomly generated $20x20$ matrices together.
\item[Details:]
  Racket's matrix library is typed, while this multiplication is untyped.
  Multiplying the two matrices generates typed/untyped boundary contracts.
\end{description}
\url{https://github.com/LeifAndersen/experimental-methods-in-pl/blob/master/exp1/examples/matrix.rkt}

\subsection*{Funky~Town~\hrulefill}
\begin{description}
\item[Source:] Soft Contracts
\item[Lines of code:] 446
\item[Overview:]
  Generates music for Lipp's Funky Town:\\
  \url{https://www.youtube.com/watch?v=xF77Y1JLScc}
\item[Details:]
  Funky~Town is a script that uses Racket's Synth library to generate a synthesized version of Funky~Town.
  While the synth library is untyped, it uses Racket's math library, which is typed.
  As with Matrix, Funky~Town generates contracts at the typed/untyped boundaries.
\end{description}
\url{https://github.com/LeifAndersen/experimental-methods-in-pl/blob/master/exp1/examples/funky-town.rkt}

\subsection*{Type~Boundary~\hrulefill}
\begin{description}
\item[Source:] Max New
\item[Lines of code:] 53
\item[Overview:]
  Recursively calls typed and untyped code.
\item[Details:]
  Type~Boundary provides two modules, one typed and one untyped, that are mutually recursive
  Like Matrix and Funky~Town, Type~Boundary will generate contracts at typed/untyped boundaries.
\end{description}
\url{https://github.com/LeifAndersen/experimental-methods-in-pl/blob/master/recursion/typed-boundary-bench.rkt}

\begin{landscape}
\pagenumbering{gobble}

\begin{figure}[htb]
{\large \textbf{Contract, JIT runtime \hspace{6cm} No Contract, JIT runtime}}
\vspace{0.1cm}

\hspace{-1.2cm}\csvautotabular{data/contracts-jit-runtime.csv}\septable\csvautotabular{data/nocontracts-jit-runtime.csv}
\end{figure}

\begin{figure}[htb]
{\large \textbf{Contract, No JIT runtime \hspace{5.2cm} No Contract, No JIT runtime}}
\vspace{0.1cm}

\hspace{-1.2cm}\csvautotabular{data/contracts-nojit-runtime.csv}\septable\csvautotabular{data/nocontracts-nojit-runtime.csv}
\end{figure}

\begin{figure}[htb]
{\large \textbf{JIT speedup with Contracts \hspace{5cm} JIT speedup with No Contracts}}
\vspace{0.1cm}

\hspace{-1.2cm}\csvautotabular{data/contracts-jit-speedup.csv}\septable\csvautotabular{data/nocontracts-jit-speedup.csv}
\end{figure}
\end{landscape}
